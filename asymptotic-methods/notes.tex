\documentclass[a4paper]{article}
% Some basic packages
\usepackage[utf8x]{inputenc}
\usepackage[T1]{fontenc}
\usepackage[english]{babel}
\usepackage{textcomp}
\usepackage{url}
\usepackage{graphicx}
\usepackage{float}
\usepackage{booktabs}
\usepackage{enumitem}
\usepackage{geometry}[margin=1in]
% Don't indent paragraphs, leave some space between them
\usepackage{parskip}

% Footnotes at bottom
\usepackage[bottom]{footmisc}

% Hide page number when page is empty
\usepackage{emptypage}
\usepackage{subcaption}
\usepackage{multicol}

% Math stuff
\usepackage{amsmath, amsfonts, mathtools, amsthm, wasysym, braket}
% Fancy script capital
\usepackage{mathrsfs}
\usepackage{cancel}
% Bold math
\usepackage{bm}
% Some shortcuts
\newcommand\N{\ensuremath{\mathbb{N}}}
\newcommand\R{\ensuremath{\mathbb{R}}}
\newcommand\Z{\ensuremath{\mathbb{Z}}}
\renewcommand\O{\ensuremath{\emptyset}}
\newcommand\Q{\ensuremath{\mathbb{Q}}}
\newcommand\C{\ensuremath{\mathbb{C}}}
\newcommand{\bigzero}{\mbox{\normalfont\Large 0}}
\renewcommand\P{\ensuremath{\mathbb{P}}}
\newcommand\E{\ensuremath{\mathbb{E}}}
\DeclareMathOperator{\dim}{dim}
\DeclareMathOperator{\intr}{int}

% Add inline-code support
\usepackage{listings}
\usepackage{color}

\definecolor{dkgreen}{rgb}{0,0.6,0}
\definecolor{gray}{rgb}{0.5,0.5,0.5}
\definecolor{mauve}{rgb}{0.58,0,0.82}

\lstset{frame=tb,
  language=Python,
  aboveskip=3mm,
  belowskip=3mm,
  showstringspaces=false,
  columns=flexible,
  basicstyle={\small\ttfamily},
  numbers=none,
  numberstyle=\tiny\color{gray},
  keywordstyle=\color{blue},
  commentstyle=\color{dkgreen},
  stringstyle=\color{mauve},
  breaklines=true,
  breakatwhitespace=true,
  tabsize=3
}

% Hyperlink support
\usepackage{hyperref}
\hypersetup{
	colorlinks,
	linkcolor= blue
}
% Put x \to \infty below \lim
\let\svlim\lim\def\lim{\svlim\limits}

%Make implies and impliedby shorter
\let\implies\Rightarrow
\let\impliedby\Leftarrow
\let\iff\Leftrightarrow
\let\epsilon\varepsilon

% Add \contra symbol to denote contradiction
\newcommand\contra{\scalebox{1.5}{$\lightning$}}

% horizontal rule
\newcommand\hr{
    \noindent\rule[0.5ex]{\linewidth}{0.5pt}
}

% hide parts
\newcommand\hide[1]{}

% Number equations from within sections
\numberwithin{equation}{section}

% Environments
\makeatother
% For box around Definition, Theorem, \ldots
\usepackage{mdframed}
\mdfsetup{skipabove=1em,skipbelow=0em}
\theoremstyle{definition}
\newmdtheoremenv[nobreak=true]{defn}[equation]{Definition}
\newmdtheoremenv[nobreak=true]{prob}[equation]{Problem}
\newmdtheoremenv[nobreak=true]{apprx}[equation]{Approximation}
\newmdtheoremenv[nobreak=true]{property}[equation]{Property}
\newmdtheoremenv[nobreak=true]{lemma}[equation]{Lemma}
\newmdtheoremenv[nobreak=true]{prop}[equation]{Proposition}
\newmdtheoremenv[nobreak=true]{thm}[equation]{Theorem}
\newmdtheoremenv[nobreak=true]{corol}[equation]{Corollary}
\newmdtheoremenv[nobreak=true]{law}[equation]{Law}
\newmdtheoremenv[nobreak=true]{nremark}[equation]{Remark}
\newmdtheoremenv[nobreak=true]{postulate}[equation]{Postulate}
\newmdtheoremenv[nobreak=true]{conc}[equation]{Conclusion}
\newtheorem*{notation}{Notation}
\newtheorem*{observation}{Observation}
\newtheorem*{ex}{Exercise}
\newtheorem*{note}{Note}
\newtheorem*{application}{Application}
\newtheorem*{eg}{Example}
\newtheorem*{remark}{Remark}
\newtheorem*{question}{Question}
\newtheorem*{observe}{Observe}
\newtheorem*{intuition}{Intuition}
\newtheorem{assumption}{Assumption}
\renewcommand*{\theassumption}{\Alph{assumption}}
\newtheorem{q}{Question}
\newtheorem*{idea}{Idea}

% End example and intermezzo environments with a small diamond (just like proof
% environments end with a small square)
\usepackage{etoolbox}
\AtEndEnvironment{vb}{\null\hfill$\diamond$}%
\AtEndEnvironment{intermezzo}{\null\hfill$\diamond$}%
% \AtEndEnvironment{opmerking}{\null\hfill$\diamond$}%

% Fix some spacing
% http://tex.stackexchange.com/questions/22119/how-can-i-change-the-spacing-before-theorems-with-amsthm
\makeatletter
\def\thm@space@setup{%
  \thm@preskip=\parskip \thm@postskip=0pt
}


% These are the fancy headers
\usepackage{fancyhdr}

% Todonotes and inline notes in fancy boxes
\usepackage{todonotes}
\usepackage{tcolorbox}

% Make boxes breakable
\tcbuselibrary{breakable}

% Figure support as explained in my blog post.
\usepackage{import}
\usepackage{xifthen}
\pdfminorversion=7
\usepackage{pdfpages}
\usepackage{transparent}
\newcommand{\incfig}[1]{%
    \def\svgwidth{\columnwidth}
    \import{./figures/}{#1.pdf_tex}
}

% Fix some stuff
% %http://tex.stackexchange.com/questions/76273/multiple-pdfs-with-page-group-included-in-a-single-page-warning
\pdfsuppresswarningpagegroup=1



\title{Asymptotic Methods}
\author{}

\begin{document}
	
\maketitle

\section{Introduction}
	
	 \[
		 f(x) = x + 3 + \frac{1}{x^2}
	 \]
	 graph showing asymptotically like x+3

	 \[
		 f(x) - (x+3) \to \frac{1}{x^2}
	 \]
	 \[
		 g(x) = x^2 + x + 3 + \frac{1}{x^{3}}
	 \] 

	 The purpose of this course is to analyse how functions, integrals, and solutions to differential equations behave in some asymptotic limit.


\subsection{Important Definitions}

\begin{defn} Big $\mathcal{O}$ Notation

	For $f: (a, \infty) \to \C / \R$, $g: (a,\infty) \to \R$, with $g(x)>0$ for  $x\ge A > a$, we say
	\[
		f(x) = \mathcal{O}\left( g(x) \right)  \text{ as } x\to \infty
	\]
	if $\exists M > 0, B>0 $  st
	\[
		|f(x)| \le Mg(x), \; x\ge B>A
	\] 
	

	We say
	\[
		f(x) =  \mathcal{O}\left( g(x) \right) x\to x_0
	\]
	if $\exists M, \delta > 0$ s.t.
	\[
		|f(x)| \le M g(x), \; 0 < |x-x_0| < \delta
	\]

	or alternatively
	\[
		\lim \sup_{x\to x_0} \frac{|f(x)|}{g(x)} < \infty
	\] 
\end{defn}

\begin{observation}
	If $f(x) = \mathcal{O}\left( g(x)  \right), x\to x_0$, then $ cf(x) =  \mathcal{O}\left( g(x) \right), c \in \R $
\end{observation}

\begin{eg}
	$f(x) = \frac{1}{x^2}\sin(\frac{1}{x}), x\to 0$ is $\mathcal{O(\frac{1}{x^2}})$, as $\lim\sup_{x\to 0} |\sin(\frac{1}{x})| = 1$
\end{eg}


\begin{defn} Little-o Notation
	
	$f(x) = o\left( g(x) \right)$ as $x\to x_0$ if
	\[
		\lim_{x\to x_0} \frac{f(x)}{g(x)} = 0
	\] 

\end{defn}

\begin{eg}
	$f(x) = \frac{1}{x^2}\sin(\frac{1}{x})$, $g(x) = \frac{1}{x^3}$

	\[
		\lim_{x\to 0} \frac{f(x)}{g(x)} = \lim_{x\to 0} x \sin\left(\frac{1}{x}\right) = 0
	\]
	So as $x\to 0$, $f(x) = o\left( g(x) \right) $
\end{eg}

\begin{notation}

	$f(x) = o\left( g(x) \right) $ is sometimes written as $f(x) \ll g(x)$.
	
\end{notation}

\section{Asymptotic Series}

\begin{defn} Asymptotic Sequence
	
	$\phi_n : D \subset \C \to  \C$, $n = 0, 1, \ldots$ is called an asymptotic sequence as $z\to z_0 \in D$, if
	\[
		\forall n > m, \phi_n(z) = o\left( \phi_m(z) \right) \text{ as } z\to z_0
	\] 
\end{defn}

\begin{eg}
	\begin{enumerate}
		\item $\phi_n(x) = x^{n-3}$ defines an asymptotic sequence as $x\to 0$.

		\item $\phi_n(x) = (x-5)^{-n}$ defines an asymptotic sequence as $x\to \infty$

		\item $\phi_n(x) = \frac{1}{x^2}\cos(nx)$ is not asymptotic as $x\to 0$, as $\lim\sup_{x\to 0} \frac{\cos(nx)}{\cos(mx)} = 1$. 
	\end{enumerate}
\end{eg}

Using simple asymptotic sequences, our goal is to describe the behaviour of much more complicated functions.

\begin{defn} Asymptotic Expansion

	Let $\phi_n : D \subset \C \to \C$ be an asymptotic sequence about $z_0$.

	We say the sum $\sum_{n=0}^{\infty} a_n \phi_n(z)$ is an asymptotic expansion of f(z) as $z\to z_0$ if for all  $N \in \N$, as $z\to z_0$,
	\[
		f(z) - \sum_{n=0}^{N} a_n \phi_n(z) = o\left( \phi_N(z) \right) 
	\] 

	i.e. 
	\[
		\lim_{z\to z_0} \frac{f(z) - \sum_{n=0}^{N}a_n \phi_n(z)}{\phi_N(z)} = 0
	\]

	If this holds, we write
	\[
		f(z) \sim \sum_{n=0}^{\infty} a_n \phi_n(z)
	\] 
\end{defn}

\begin{remark}
	We do not require that $\sum_{n=0}^{\infty} a_n \phi_n(z)$ converges.
\end{remark}

\begin{prop}
	If $f(z) \sim \sum_{n=0}^{\infty} a_n \phi_n(z)$, then
	\[
		a_{N+1} = \lim_{z\to z_0} \frac{f(z) - \sum_{n=0}^{N} a_n \phi_n}{\phi_{N+1}}
	\] 
\end{prop}

\begin{remark}
	If $a_n =0 \forall n$, we have that $\lim_{z\to z_0} \frac{f(z)}{\phi_n(x)} = 0$, so our asymptotic sequence is subdominated and provides no information.
\end{remark}

\begin{eg} (Taylor)
	
	$f \in C^{\infty}: [a,b] \to \R$, $f(x) \sim \sum_{n=0}^{\infty} \frac{f^{(n)}(x_0)}{n!} (x-x_0)^{n}$

	We have that $f(x) - \sum_{n=0}^{N} \frac{f^{(n)}(x_0)}{n!} (x-x_0)^{n} = \int_{x_0}^{x} \frac{(x-t)^{N}}{N!} f^{(N+1)}(t) dt =: R_N(x)$.

	$|R_N(x)| \le \max_{a\le t\le b} |f^{(N+1)}(t)| \frac{1}{N!}\int_{x_0}^{x} |x-t|^{N} = \frac{|x-x_0|^{N+1}}{(N+1)!} \max |f^{(N+1)}(t)|$

	So $\left|\frac{R_N(x)}{(x-x_0)^{N}} \right| \le \frac{|x-x_0|}{(N+1)!} \max |f^{(N+1)}(t)| \to 0$ 
\end{eg}

\begin{prop}
	\begin{enumerate}
		\item If $f(z) \sim \sum_{n=0}^{\infty} a_n \phi_n(z)$, $g(z) \sim \sum_{n=0}^{\infty} b_n \phi_n(z)$, then
			\[
				\alpha f + \beta g \sim \sum_{n=0}^{\infty} (\alpha a_n+ \beta b_n) \phi_n(z)
			\]

		\item If $f(z) \sim \sum_{n=0}^{\infty} a_n (x-x_0)^{n}$, $g(z) \sim \sum_{n=0}^{\infty} b_n (z-z_0)^{n}$ then
			\[
				f\cdot g (z) \sim \sum_{n=0}^{\infty} c_n (z-z_0)^{n}
		\]
		with $c_n = \sum_{k=0}^{n} a_k b_{n-k}$


	\end{enumerate}
\end{prop}


\begin{prop}
	
	The asymptotic expansion is unique. That is to say, if $f(z) \sim \sum_{n=0}^{\infty} a_n \phi_n(z)$, $f(z) \sim  \sum_{n=0}^{\infty} b_n \phi_n(z)$, then $a_n = b_n$ everywhere.
\end{prop}

\begin{proof}
	\begin{align*}
		a_0 - b_0 = \frac{(a_0\phi_0 - f) + (f - b_0 \phi_0)}{\phi_0} \to 0
	\end{align*}

	The rest follows inductively.
\end{proof}

\begin{eg}
	$f(x) = \begin{cases}
	 e^{-x^2}, x\neq 0 \\
	 0, x=0
	\end{cases}$

	Then $f(x)$ subdominates all polynomials, so $f(x) + \sin(x)$ has the same asymptotic expansion as $\sin(x)$ about $0$. Hence we can see that while the asymptotic expansion of a function is unique, an asymptotic expansion does not uniquely define a function. 
\end{eg}

\begin{prop}

	If $f(x) \sim \sum_{n=0}^{\infty} a_n (x-x_0)^{n}$, then
	\[
		\int_{x_0}^{x} f(\xi) d\xi \sim \sum_{n=0}^{\infty} \frac{a_n}{n+1} (x-x_0)^{n+1}
	\] 
\end{prop}

\begin{proof}
	Let $\epsilon > 0$.  $\exists \delta(\epsilon) > 0$ s.t. if $0<|\xi - x_0| < \delta$, then
	\[
		\left| f(\xi) - \sum_{n=0}^{N} a_n (\xi - x_0)^{n} \right| \le \epsilon |\xi - x_0|^{N}
	\]
	Then integrating,
	\begin{align*}
		\left|\int_{x_0}^{x} f(\xi) - \sum_{n=0}^{N} a_n (\xi - x_0)^{n} d\xi \right| &\le \int_{x_0}^{x} \left| f(\xi) - \sum_{n=0}^{N} a_n (\xi - x_0)^{n}\right| d\xi \\
		&\le  \epsilon \int_{x_0}^{x} |\xi - x_0|^{N} d\xi \\
		&\le \epsilon \frac{|x-x_0|^{N+1}}{N+1}
	\end{align*}
	And hence
	\[
		\frac{\left| \int_{x_0}^{x} f(\xi)d\xi - \sum_{n=0}^{N} \frac{a_n (x-x_0)^{n+1}}{n+1}\right|}{|x-x_0|^{N+1}} \le \epsilon
	,\]
	i.e. \[
		\int_{x_0}^{x} f(\xi) d\xi - \sum_{n=0}^{N} a_n (x-x_0)^{n+1} = o\left( (x-x_0)^{N+1} \right) 
	\] 

\end{proof}

\begin{ex}
	For $x_0 \in  (a, b)$, and $f(x)$ continuous on $[a,b]$, let $\{\phi_n\}\rvert_{n=0}^{\infty} $ be an asymptotic sequence of functions on $[a,b]$ s.t. $\phi_n(x) \neq 0$ for $x\neq x_0$. Then,
	\begin{enumerate}
		\item $\int_{x_0}^{x} \phi_n(\xi) d\xi$ is an asymptotic sequence. \\
		\item $\int_{x_0}^{x} f(\xi)d\xi \sim \sum_{n=0}^{\infty} a_n \int_{x_0}^{x} \phi_n(\xi) d\xi$
	\end{enumerate}
\end{ex}

\begin{ex}
	$f(x) = \tan x, x=0$.

	$f(x) = x + \frac{x^3}{3} + \ldots = \sum_{n=1}^{\infty}  \frac{2^{n} (2^{n-1}}{(2n)!} |B_{2n}| x^{2n-1}$

	$\phi_n(x) = (\sin x)^{n} ~x^{n}$ 

	Then we have asymptotic expansion
	\[
		\tan x \sim \sum_{n=1}^{\infty} \frac{(2n)!}{(n!)^{2}} (\sin x)^{n}
	\] 
\end{ex}

\subsection{Asymptotic Integrals}

\subsection*{Gamma Function}

\begin{align*}
	\Gamma(z) &= \int_{0}^{\infty} t^{z-1} e^{-t} dt\ \quad Re(z) > 0 \\
	&= \frac{t^{z}}{z} e^{-t} \big\rvert_{0}^{\infty} \int_{0}^{\infty} \frac{t^{z}}{z} e^{-t} dt \\
	\implies \Gamma(z) &= \frac{1}{z} \Gamma(z+1)
\end{align*}
Noting that $\Gamma(1) = \int_{0}^{\infty} e^{-t} dt =1$, we have that $n! = \Gamma(n+1)$.


$\Gamma(n+1) = n! \sim (2\pi n)^{\frac{1}{2}} (\frac{n}{e})^{n}$ 

$\Gamma(n) \sim (\frac{2\pi}{n})^{\frac{1}{2}} (\frac{n}{e})^{n}[n+\frac{1}{12n} + \frac{1}{28 n^2} + \ldots] $

$\Gamma(\frac{1}{2}) = \int_{0}^{\infty} t^{-\frac{1}{2}} e^{-t} dt = \int_{0}^{\infty} 2 e^{-s^2} ds = \sqrt{\pi} $

\subsection*{Stilljes Integral}

$S(x) = \int_{0}^{\infty} \frac{\rho(t)}{1+xt} dt,\quad x\to 0^{+}$.
Here, $\rho(t)$ is very smooth, and decays very fast at $\infty$.

We have that $S(0) = \lim_{x\to 0} S(x) = \int_{0}^{\infty} \rho(t) dt$. 

Now, \begin{align*}\frac{1}{1+xt} &= \sum_{n=0}^{N} (-xt)^{n} + \frac{(-1)^{N+1} x^{N+1} t^{N+1}}{1+xt} \\
	\frac{1}{1+xt} - \sum_{n=0}^{N} (-xt)^{n} &= \frac{(-1)^{N+1} x^{N+1} t^{N+1}}{1+x
t} \\
\implies \int_{0}^{\infty} \frac{\rho(t)}{1+xt}dt - \sum_{n=0}^{N} \int_{0}^{\infty} (-xt)^{n}\rho(t)dt &= (-x)^{N+1} \int_{0}^{\infty} \frac{\rho(t) t^{N+1}}{1+xt} \\	
\end{align*}
Let $C_n = \int_{0}^{\infty} t^{n} \rho(t) dt$ be the moments of $\rho(t)$, and note that  $\frac{1}{1+xt} \le 1$ for $x,t > 0$.

Then  \[
	\left| \int_{0}^{\infty}\frac{\rho(t)}{1+xt}dt - \sum_{n=0}^{N}(-1)^{n} C_n x^{n} \right| \le  x^{N+1} C^{N+1} = o(x^{N}) \text{ as } x\to 0
\] 

Hence
\[
	\int_{0}^{\infty} \frac{\rho(t)}{1+xt} dt \sim \sum (-1)^{n} C_n x^{n}
\] 

\subsection{Approximation of Integrals}

\begin{thm}
	Let $f, g$ be continuously differentiable, with integrable derivatives on $(a,b)$. Then
	 \[
		 \int_{a}^{b} f'(t) g(t) dt = \left[ f(t) g(t) \right]_{a}^{b} - \int_{a}^{b} f(t) g'(t) dt 
	\]

	In general,
	\[
	\int_{a}^{b} f^{(n)}(t) g(t)dt = \sum_{k=1}^{n} (-1)^{k-1} f^{(n-k)}(t)g^{(k)}(t) \big\rvert_{a}^{b} + (-1)^{n} \int_{a}^{b} f(t) g^{(n)}(t)dt
	\] 
\end{thm}

 \begin{eg}
	 $\erf(x) = \frac{1}{\sqrt{2\pi}} \int_{-\infty} ^{x} e^{-\frac{t^2}{2}} dt$. We wish to consider its behaviour for $x \gg 1$.

	 Let $F(x) = \sqrt{2\pi} (1-\erf(x) )$. Then
	 \begin{align*}
		 F(x) &= \int_{x}^{\infty} e^{-\frac{t^2}{2}} \frac{t}{t} dt \\
		 &= -\frac{1}{t} e^{-\frac{t^2}{2}} \big\rvert_{x}^{\infty} - \int_{x}^{\infty}\frac{1}{t^2} e^{-\frac{t^2}{2}} dt \\
		 &= \frac{e^{-\frac{x^2}{2}}}{x} - \int_{x}^{\infty} \frac{1}{t^2} e^{-\frac{t^2}{2}}dt \\
		 &:= \frac{e^{-\frac{x^2}{2}}}{x} -R(x) \\
		 |R(x)| &= \int_{x}^{\infty} \frac{e^{-\frac{t^2}{2}}}{t^2} \frac{t}{t}dt \le \frac{1}{x^{3}} e^{-\frac{x^2}{2}} \text{ as } \frac{1}{t^3} \le \frac{1}{x^3} \\
	 \end{align*}
	 Hence
	 \[
	 \left| \frac{F(x) - \frac{e^{-\frac{x^2}{2}}}{x}}{\frac{e^{-\frac{x^2}{2}}}{x}} \right| \le  \frac{1}{x^2} \text{ as } x \to \infty
	 \] 

	 Repeating inductively, integrating by parts, we get that

	 \[
		 \int_{x}^{\infty} e^{-\frac{t^2}{2}} dt \sim \frac{e^{-\frac{x^2}{2}}}{x} \sum_{n=0}^{\infty} \frac{(-1)^{n} (2n-1)!!}{x^{2n}}
	 \] 
\end{eg}

\begin{eg}
	Let $f \in C^{\infty}([a,b])$, with $f(b) \neq 0$. Find the expansion of 
	\[
		I(x) = \int_{a}^{b} f(t) e^{xt} dt \text{ as } x \to \infty
	.\]

	Intuition: Near $t=b$, is the main contribution. So we expect
	 \[
		 \int_{a}^{b} f(t) e^{xt} dt \sim \int_{b-\epsilon}^{b} f(t)e^{xt}dt \sim  f(b) \int_{b-\epsilon}^{b} e^{xt}dt = f(b) \frac{e^{xb} - e^{x(b-\epsilon)}}{x} \sim \frac{f(b) e^{xt}}{x}
	\]

	Integrating by parts, it is easy to see that
	\begin{align*}
	I(x) &= \sum_{k=1}^{n} f^{(n-1)} \frac{e^{xt}}{x^{k}}(-1)^{k-1} \Big\rvert_{a}^{b} + (-1)^{n} \int_{a}^{b} f^{(n-1)}(t) \frac{e^{xt}}{x^{n}} dt \\
	&=  \sum_{k=1}^{n} f^{(n-1)}(b) \frac{e^{xb}}{x^{k}}(-1)^{k-1} + (-1)^{n} \int_{a}^{b} f^{(n-1)}(t) \frac{e^{xt}}{x^{n}} dt -\sum_{k=1}^{n} f^{(n-1)}(a) \frac{e^{xa}}{x^{k}}(-1)^{k-1} \\
	&= e^{xb}\left( \sum_{k=1}^{n} f^{(n-1)}(b) \frac{1}{x^{k}}(-1)^{k-1} \Big\rvert_{a}^{b} + (-1)^{n} \int_{a}^{b} f^{(n-1)}(t) \frac{e^{x(t-b)}}{x^{n}} dt -\sum_{k=1}^{n} f^{(n-1)}(a) \frac{e^{-x(b-a)}}{x^{k}}(-1)^{k-1} \right) \\
\end{align*} 

We can see that $e^{-x(b-a)} = o(x^{-m}, m = 1,2,\ldots$

\begin{align*}
	\left| (-1)^{n} \int_{a}^{b} f^{(n-1)}(t) \frac{e^{x(t-b)}}{x^{n}} \right| &\le \max_{a\le t\le b} |f^{(n-1)}(t)| \int_{a}^{b} \frac{e^{x(t-b)}}{x^{n}} dt \\
	&= \frac{1}{x^{n+1}}(e^{-x(b-a)} - 1) \underbrace{\max_{a\le t \le b} |f^{(n-1)}(t)|}_{:=C_n} \\
	&= o\left(\frac{1}{x^{n}}\right)
\end{align*}

We can then see that
\[
	I(x) \sim  e^{xb} \sum_{k=1}^{n}\frac{(-1)^{k-1} f^{(k-1)}(b)}{x^{k}} 
,\] so in fact
\[
	I(x) \sim \frac{f(b)}{x} e^{xb}
\] 
\end{eg}

\begin{thm} Watson's Lemma

	Let $0 < T \le \infty $. Suppose f(t) has an asymptotic expansion about $t=0$ ( $t>0$)
	 \[
		 f(t) \sim t^{\alpha} \sum_{n=0}^{\infty} a_n t^{\beta n} \text{ as } t \to 0^{+}
	 ,\] with $\alpha > -1$,  $\beta > 0$.

	 Assume further that $f(t)$ satisfies one of the following:

	 \begin{enumerate}
		\item $|f(t)| \le  K e^{bt}, \quad t\ge 0$ \\
		\item $\int_{0}^{T} |f(t)| dt < \infty$	
	 \end{enumerate}
	
	 Then
	 \[
		 F(x) = \int_{0}^{T} e^{-xt}f(t) dt \sim \sum_{n=0}^{\infty} \frac{a_n \Gamma(\alpha + \beta n + 1)}{x^{\alpha + \beta n + 1}} \text{ as } x\to \infty
	 \]

\end{thm}

\begin{proof}
	\[
		F(x) = \int_{o}^{T} e^{-xt}f(t) dt = \int_{0}^{\epsilon}e^{-xt} f(t) dt + \int_{\epsilon}^{T} e^{-xt}f(t) dt
	\]

	Recall that  \[
		\int_{0}^{\infty} e^{-xt} t^{\lambda} dt \overset{u=xt}{=} \int_{0}^{\infty} e^{-u} \frac{u^{\lambda}}{x^{\lambda}} \frac{du}{x} = \frac{\Gamma(\lambda + 1)}{x^{\lambda + 1}}
	\] 

	We need to show that 
	\[
		R_N(x) := F(x) - \sum_{n=0}^{N} \frac{a_n \Gamma(\alpha + \beta n + 1)}{x^{\alpha + \beta n + 1}} = o\left( \frac{1}{x^{\alpha + \beta N+ 1}} \right) \text{ as } x\to \infty
	\]

	By our above comment,
	\[
		R(x) = F(x) - \sum_{n=0}^{N} a_n \int_{0}^{\infty} e^{-xt} t^{\alpha + \beta n} dt
	\]

	We can rewrite this as
         \begin{align*}
		 R_N(x) &= \int_{0}^{\epsilon} f(t) e^{-xt}dt + \int_{\epsilon}^{T} f(t) e^{-xt} dt - \sum_{n-0}^{N}a_n \left( \int_{0}^{\epsilon} e^{-xt}t^{\alpha + \beta n +1} dt + \int_{\epsilon}^{T} e^{-xt} t^{\alpha + \beta n} \right) \\
		 &= \left( \int_{0}^{\epsilon} e^{-xt} \left(f(t) - \sum_{n=0}^{N} a_n t^{\alpha+\beta n}\right) dt \right) + \int_{\epsilon}^{T} e^{-xt}f(t)dt + \sum_{n=0}^{N} a_n \int_{\epsilon}^{T} e^{-xt}t^{\alpha + \beta n} dt \\
		 &:= R_{N_1} + R_{N_2} + R_{N_3}
         \end{align*}

	 Now,
	 \begin{align*}
		 |R_{N_3}| &\le \sum_{n=0}^{N} |a_n| \int_{\epsilon}^{T} e^{-xt} t^{\alpha + beta n}dt \\
		 &= e^{-\epsilon x} \sum_{n=0}^{N} \int_{\epsilon}^{T} e^{-x(t-\epsilon)} t^{\alpha + \beta n} \\
		 &\le  \frac{e^{-\epsilon x}}{x} \sum_{n=0}^{N} |a_n| \int_{0}^{\infty} e^{-u} \left(1+\frac{u}{\epsilon x}\right)^{\alpha + \beta n} du \; \epsilon^{\alpha + \beta n} \text{ using $u = xt$}\\
		 &\le \frac{e^{-\epsilon x}}{x} \sum_{n=0}^{N} |a_n| \int_{0}^{\infty} e^{-u} \left(1+\frac{u}{\epsilon x}\right)^{\alpha + \beta n} du \; \epsilon^{\alpha} \text{ as $\beta$ positive} \\
		 &\le \frac{e^{-\epsilon x} \epsilon^{\alpha}}{x} \sum_{n=0}^{N}|a_n| \int_{0}^{\infty} C(\alpha, \beta, N) \left( 1 + \left( \frac{u}{\epsilon x} \right)^{\alpha + \beta n + 1}  \right)du \\
		 &=o(x^{-m}) \; \forall m\ge 1
	 \end{align*}

	Next,
	\begin{align*}
		|R_{N_1}| &= \left|\int_{0}^{\epsilon}e^{-xt} \left( f(t) - \sum_{n=0}^{N} a_n t^{\alpha + \beta N} \right) dt \right| \\
		&\le \int_0^{\epsilon} e^{-xt} \eta t^{\alpha + \beta N} dt \\
		&\le \eta \int_0^{\infty} e^{-xt} t^{\alpha + \beta n}dt \text{ for some suitably chosen $\eta(\epsilon)$}\\
		&= \eta \frac{\Gamma(\alpha + \beta n + 1}{x^{\alpha + \beta n + 1}} \\
		&= o\left( x^{-(\alpha + \beta n + 1)} \right) 
	\end{align*} 

	In the case that $|f(t)|$ is integrable,

	\begin{align*}
		|R_{N_2}| &\le e^{-xt} |f(t)| dt \\
			&\le  e^{-\epsilon x} \int_{0}^{T}|f(t)| dt
			&= o(x^{-m}) \; \forall m \ge_1
	\end{align*}

	Finally, in the case that $|f(t)| \le  Ke^{bt}$ asymptotically,
	\begin{align*}
		|R_{N_2}| &\le \int_{\epsilon}^{T} e^{-xt} |f(t)| dt \\
		&\le K \int_{\epsilon}^{T} e^{-xt}e^{bt} dt \text{ by assumption} \\
		&\le K \frac{e^{-(x-b)\epsilon}}{x-b} \\
		&= o(x^{-m}) \; \forall m \gg 1
	\end{align*}
\end{proof}

If we instead consider complex values,
\begin{align*}
	F(z) &= \int_0^{T} e^{-zt} f(t) dt
\end{align*}
Now, \[
|e^{-zt}| = e^{-xt}
,\]
and $|x|\ge |z| - |y| \ge |z|(1-\sin \theta) \ge |z|(1 - \sin\delta)$, where we enforce that $z$ is at least an angle $\delta$ into the RHP, $0<\delta<\frac{\pi}{2}$.

Like this, we can reformulate Watson's lemma for complex values.

\begin{eg}
	$F(x) = \int_{0}^{\infty} e^{-xt} \sin t dt$.

	$f(t) = \sin t = \sum_{k=0}^{\infty} \frac{(-1)^{k}t^{2k+1}}{(2k+1)!}$, $|f(t)| \le 1$, so we can apply Watson's lemma.

	Then \[
		I(x) \sim  \sum_{k=0}^{\infty} \frac{(-1)^{k}}{(2k+1)!} \frac{\Gamma(2k+1 + 1)}{x^{2k+2}}
	\]
	or more neatly, as $k \in \N$,
	\[
		I(x) \sim \frac{1}{x^2} \sum_{k=0}^{\infty} \frac{(-1)^{k}}{x^{2k}}
	\] 
\end{eg}

\begin{eg}
	One of the solutions of the hypergeometric equation
	\[
		xy_{x x} + (b-x) y_x - ay = 0
	\]
	With $\Re b > \Re a > 0$, is given by
	 \[
		 M(a,b,x) = \frac{e^{x}}{\Gamma(a)\Gamma(b-a)} \int_{0}^{1}  e^{-xt} (1-t)^{1-a} t^{b-a-1} dt
	\]

	So, let $f(t) = (1-t)^{1-a}t^{b-a-1}, \quad 0\le t\le 1$.
	\begin{align*}
		f(t) = t^{b-a-1} \sum_{n=0}^{\infty} \frac{(-1)^{n}}{n!} \frac{\Gamma(a)}{\Gamma(a-n)} t^{n}
	\end{align*}
	and as such we can apply Watson's Lemma to $I(x)$, getting 
	\[
		M(a,b,t) = \sum_{n=0}^{\infty} \left( \frac{(-1)^{n} \Gamma(a)}{\Gamma(a-n) n!} \frac{\Gamma(b-a+n)}{z^{b+a-n}} \right) \frac{e^{x}}{\Gamma(a)\Gamma(b-a)}
-a)
\]

So to leading order, \[
	M(a,b,x) = \frac{e^{x}}{x^{b-a} \Gamma(a)}
.\] 
\end{eg}

\section{Asymptotics of Integrals}

\subsection{Laplace's Method}

Consider
\[
	F(x) = \int_a^{b} f(t) e^{x\phi(t)} dt
,\] as $x\to \infty$. 

Firstly, consider the case that $\phi$ has a \textbf{unique} global maximum at $c\in [a,b]$, such that $\phi'(c) = 0$, and  $\phi''(c) < 0$. Assume also that $f(c) \neq 0$

Then we have  
\begin{align*}
	F(x) &\sim \int_{c-\epsilon}^{c+\epsilon} f(t) e^{x\phi(t)}dt \\
	&\sim \int_{c-\epsilon}^{c+\epsilon} f(t) e^{x (\phi(c) + \frac{1}{2} \phi''(c) (t-c)^2)} dt \\
	&\sim f(c)e^{x\phi(c)} \int_{c-\epsilon}^{c+\epsilon}e^{\frac{1}{2}\phi''(c)(t-c)^2}dt \\
	&\sim \frac{f(c) e^{x\phi(c)}}{|x\phi''(c)|^{\frac{1}{2}}} \underbrace{\int_{-|x\phi''(c)|^{\frac{1}{2}}}^{|x\phi''(c)|^{\frac{1}{2}}} e^{-\frac{s^2}{2}}ds}_{s= \sqrt{-x\phi''(c)}(t-c)}\\
	&\sim \frac{\sqrt{2\pi} }{|x\phi''(c)|^{\frac{1}{2}}} e^{x\phi(c)}f(c)
\end{align*}

\begin{eg}
	\[
		I(x) = \int_{-\infty}^{\infty} e^{-xt^2} e^{at} dt
	\]

	So we have $f(t) = e^{at}$, $\phi(t) = -t^2$, and
	\begin{align*}
		I(x) &\sim \frac{e^{a\cdot 0}e^{0} \sqrt{2\pi} }{(2x)^{\frac{1}{2}}} \\
			&\sim \sqrt{\frac{\pi}{x}} \text{ as } x\to \infty 
	\end{align*}

	It is also quite simple to calculate this explicitly, where we find that this is as expected correct to leading order.
\end{eg}

Next, we consider the case of Laplace's method when the global maximum occurs at the endpoint $t=a$, with $\phi'(a), \, f(a) \neq 0$ (so $\phi'(a) < 0$).
 \[
	 F(x) \sim \int_{a}^{a+\epsilon} f(t) e^{x\phi(t)} dx
\]

We see $\phi(t) \sim \phi(a) + \phi'(a) (t-a)$, so that
\begin{align*}
	F(x) &\sim  \int_{a}^{a+\epsilon} f(t) e^{x\left(\phi(a) + \phi'(a)(t-a)  \right) } dt \\
	&\sim f(a)e^{x\phi(a)} \int_{a}^{a+\epsilon} e^{\phi'(a)(t-a)}dt\\
	&\sim f(a)e^{x\phi(a)} \int_0^{\epsilon x |\phi'(a)|} e^{-u} \frac{du}{x |\phi'(a)|} \\
	&\sim  \frac{f(a)e^{x\phi(a)}}{x|\phi'(a)|}\left( \underbrace{\int_{0}^{\infty} e^{-u} du}_{=1} - \underbrace{\int_{\epsilon x|\phi'(a)|}^{\infty} e^{-u} du}_{\text{exponentially decays in $x$}} \right) \\
	&\sim \frac{f(a)e^{x\phi(a)}}{x|\phi'(a)|}
\end{align*}

\begin{remark}
	If our global maximum is instead at $t=b$, with $\phi'(b) >0$, we see that
	 \[
		 F(x) \sim \frac{f(b)e^{x\phi(b)}}{x\phi'(b)}
	\] 
\end{remark}

Our third case is that we have a unique global maximum of $\phi(t)$ at $c \in (a,b)$, with $\phi'(c) = \phi''(c) =0$. Assume  $f(c) \neq 0$.

\[
	F(x) = \int_{a}^{b} f(t) e^{x\phi(t)} dt \sim \int_{c-\epsilon}^{c+\epsilon} f(t) e^{\phi(t)}dt
\]

As $c$ is a maximum, we must have that
\[
	\phi(t) \sim \phi(c) + \frac{\phi^{(p)}(c)}{p!}(t-c)^{p}
,\] where p is the first (even) power such that $\phi^{(p)}(c) \neq 0$. For a maxima, $\phi^{(p)}(c) < 0$

Then 
\begin{align*}
	F(x) &\sim \int_{c-\epsilon}^{c+\epsilon} f(c) e^{x\left( \phi(c) + \frac{\phi^{(p)}(c)}{p!}(t-c)^{p} \right) } dt  \\
	&\sim f(c)e^{x\phi(c)} \int_{c-\epsilon}^{c+\epsilon} e^{x\phi^{(p)}(c)}\frac{(t-c)^{p}}{p!} dt
\end{align*}

Let $s = \left( \frac{x |\phi^{(p)}(c)|}{p!} \right)$. So
\begin{align*}
	F(x) &\sim  f(c)e^{x\phi(c)} \int_{- \epsilon \left( \frac{x|\phi^{(p)}(c)|}{p!} \right)^{\frac{1}{p}} }^{\epsilon \left( \frac{x|\phi^{(p)}(c)|}{p!} \right)^{\frac{1}{p}} } e^{-s^{p}} \frac{1}{\left( \frac{x|\phi^{(p)}(c)|}{p!} \right)^{\frac{1}{p}} } ds  \\
	&\sim \frac{f(c)e^{x\phi(c)}}{\left( \frac{x|\phi^{(p)}(c)|}{p!} \right)^{\frac{1}{p}} } \int_{-\infty}^{\infty}e^{-s^{p}} ds \\
	&\sim \frac{2 f(c) e^{x\phi(c)} (p!)^{\frac{1}{p} \Gamma(1+p)}}{(x|\phi^{(p)}(c)|)^{\frac{1}{p}}}
\end{align*}
Where the last integral (for even $p $) is left as an exercise to the reader.

\subsection*{Laplace Method - Higher Order Expansion}

Motivation: What if $f(c) = 0$?

Again, we assume that  $\phi(t)$ has a unique global maximum at $t=c$, $c\in (a,b)$, with $\phi'(c) = 0$,  $\phi''(c) < 0$. 

So $f(c)\sim f(c) + f'(c)(t-c) + \frac{1}{2} f''(c) (t-c)^2 + \ldots$

And $\phi(c) \sim \phi(c) + \frac{1}{2} \phi''(c) (t-c)^2 + \frac{1}{6}\phi'''(c) (t-c)^3 + \frac{1}{24} \phi''''(c) (t-c)^{4} + \ldots$

Then as before,
\begin{align*}
	F(x) &\sim \int_{c-\epsilon}^{c+\epsilon} f(t)e^{x\phi(t)}dt \\
	&\sim  \int_{c-\epsilon}^{c+\epsilon} \left(f(c) + f'(c)(t-c) + \frac{1}{2} f''(c) (t-c)^2 \right)  e^{x\left(\phi(c) +\frac{1}{2} \phi''(c) (t-c)^2 + \frac{1}{6}\phi'''(c) (t-c)^3 + \frac{1}{24} \phi''''(c) (t-c)^{4} \right)} dt \\
	&\sim \int_{c-\epsilon}^{c+\epsilon} \left(f(c) + f'(c)(t-c) + \frac{1}{2} f''(c) (t-c)^2 \right) e^{x\phi(c)} e^{\frac{1}{2}\phi''(c) (t-c)^2} e^{x\left(\frac{1}{6}\phi'''(c) (t-c)^3 + \frac{1}{24} \phi''''(c) (t-c)^{4}\right)} dt \\
	&\sim \frac{e^{x\phi(c)}}{\left( |x\phi''(c)| \right)^{\frac{1}{2}} } \int_{-\epsilon|x\phi''(c)|^{\frac{1}{2}}}^{\epsilon |x\phi''(c)|^{\frac{1}{2}}} E_1(s) e^{-\frac{s^2}{2}} \exp\left( \frac{\phi'''(c) s^3}{6 x^{\frac{1}{2}} |\phi''(c)|^{\frac{3}{2}}} + \frac{\phi''''(c)}{24} \frac{s^{4}}{x|\phi''(c)|^2} \right) ds
\end{align*}

Where \[
	E_1(s) = f(c) + \frac{f'(c)}{\left( x|\phi''(c)| \right)^{\frac{1}{2}} }s + \frac{f''(c) s^2}{2x|\phi''(c)|} 
\]

Now, writing our second exponential as $e^{z}$ and expanding to second order,

\begin{align*}
	F(x) &\sim \frac{e^{x\phi(c)}}{\left( |x\phi''(c)| \right)^{\frac{1}{2}} } \int_{-\epsilon|x\phi''(c)|^{\frac{1}{2}}}^{\epsilon |x\phi''(c)|^{\frac{1}{2}}} E_1(s) E_2(s) ds 
\end{align*}

Where \[
	E_2(s) = 1 + \left[ \frac{\phi''(c) s^3}{6x^{\frac{1}{2}} |\phi''(c)|^{\frac{3}{2}}} + \frac{\phi''''(c) s^{4}}{24 z|\phi''(c)|^2 } \right]  + \frac{(\phi'''(c))^2s^{6}}{2\cdot 32 |\phi''(c)|^3}
\]

Note that if $c$ is at an endpoint, our odd terms contribute, but here, we need only consider even terms by symmetry.

So,
\begin{align*}
	F(x) &\sim \frac{e^{x\phi(c)}}{\left( |x\phi''(c)| \right)^{\frac{1}{2}} } \int_{-\epsilon|x\phi''(c)|^{\frac{1}{2}}}^{\epsilon |x\phi''(c)|^{\frac{1}{2}}} \left( f(c) + \frac{1}{x} \left[ \frac{f''(c) ^2}{2 |\phi''(c)|} + \frac{f'(c) \phi'''(c) s^{4}}{6|\phi''(c)|^2} + \frac{f(c) \phi''''(c) s^{4}}{24|\phi''(c)|^2} + \frac{f(c)(\phi'''(c))^2 s^{6}}{721 |\phi''(c)|^3}\right]  \right) ds \\
	&\sim \frac{e^{x\phi(c)}}{\left( x|\phi''(c)| \right)^{\frac{1}{2}} } \sqrt{2\pi}  \left( f(c) + \frac{1}{x} \left[ \frac{f''(c) ^2}{2 |\phi''(c)|} + \frac{f'(c) \phi'''(c) s^{4}}{2|\phi''(c)|^2} + \frac{f(c) \phi''''(c) s^{4}}{8|\phi''(c)|^2} + \frac{f(c)(\phi'''(c))^2 s^{6}}{24 |\phi''(c)|^3}\right]  \right) 
\end{align*}

Where we used the identity (exercise) that for $p=2n$ even,
\[
	\int_{-\infty}^{\infty} e^{-\frac{s^2}{2}} s^{p} ds = \sqrt{2\pi} s^{n}\Gamma(n+\frac{1}{2}) 
\] 

\section{Oscillatory Integrals}

We wish to consider integrals of the form
\[
	I(\omega) = \int_{a}^{b} f(t) e^{i\omega \phi(t)} dt
,\] as $|\omega| \to  \infty$, for $f, \phi \in \R$.

We often call $\phi(t)$ the phase function.

\begin{lemma} Riemann-Lebesgue Lemma
	
	Let $-\infty \le a \le b \le  \infty$, with $|f(t)|$ integrable.

	Then
	\[
		I(\omega) = \int_{a}^{b} f(t) e^{i\omega t} dt \to 0 \text{ as } \omega \to \infty 
	\] 
\end{lemma}

\begin{proof}
	Consider first the case where $f \in C^{1}[a,b]$ 

	\begin{align*}
		I(\omega) &= \int_a^{b} f(t)e^{i\omega t} dt \\
			  &= \frac{f(t)e^{i\omega t}}{i\omega} \Bigg\rvert_{a}^{b} - \frac{1}{i\omega} \int_{a}^{b} f'(t) e^{i\omega t}dt \\
			  |I(\omega)| &\le  \frac{|f(b)| + |f(a)|}{|\omega|} + \frac{1}{|\omega|}(b-a) \max_{a\le t\le b} |f'(t)| \\
			  &\to 0 \text{ as } |\omega| \to \infty
	\end{align*}

	In our general case, we go back to definitions of Riemann Integrability. 
	$\forall \epsilon > 0$ , there is a partition such that our lower and upper sums satisfying 
	\[ 
		m(t) \le f(t) \le M(t)
	\]
	with
	\begin{align*}
		\left|\int_{a}^{b} f(t) - m(t) dt \right| &\le \epsilon \\
		\left| \int_a^{b} M(t) - f(t)  dt \right| &\le \epsilon \\
		\left| \int_a^{b} M(t) - m(t) dt \right| &\le \epsilon
	\end{align*}

	Hence
	\begin{align*}
		|\int_a^{b} f(t)e^{i\omega t}dt| &\le  \left|\int_a^{b} \left(f(t) - m(t)\right) e^{i\omega t} dt\right| + \left| m(t) e^{i\omega t} dt \right| \\
						 &\le  \int_a^{b}f(t) - m(t) dt + \left|\int_a^{b} m(t) e^{i\omega t} dt \right| \\
						 &= \epsilon + \left|\int_a^{b} m(t)e^{i\omega t} dt \right|
	\end{align*}

	Taking $\lim\sup_{\omega\to \infty}$ of both sides, we get the result.

	Note that in our first case, if $f'$ is Riemann integrable, we get that $I(\omega) = \mathcal{O}(\frac{1}{\omega})$.
\end{proof}

\subsection{Method of Stationary Phase}

\begin{prop}
	Consider $f \in C^{\infty}[a,b]$ Then

	\begin{align*}
		I(\omega) \sum_{n=0}^{\infty} \frac{(-1)^{n}}{(i\omega)^{n+1}} \left( e^{i\omega b} f^{(n)}(b) - e^{i\omega a} f^{(n)}(a) \right) 
	\end{align*}
\end{prop}

\begin{ex}
	For $f(t) = \frac{1}{1+t}$, $I(\omega) = \int_0^{1} \frac{e^{i\omega t}}{ 1+t} dt$.

	Then  \[
		f^{(n)} (t) = \frac{(-1)^{n}n!}{(1+t)^{n+1}}
	\]
	and
	\[
		I(\omega) \sim \sum_{n=0}^{\infty} \frac{n!}{(i\omega)^{n+1}} \left[ \frac{e^{i\omega}}{2^{n+1}} - 1 \right] 
	\] 
\end{ex}

\begin{prop}
	Suppose $\phi'(t) \neq 0$ on $[a,b]$, and $f'(t)$ integrable.

	Then
	\[
		I(\omega) \sim  \frac{1}{i\omega} \left[ \frac{f(b)}{\phi'(b)} e^{i\omega \phi(b)} - \frac{f(a)}{\phi'(a)} e^{i\omega \phi(a)} \right] 
	\] 
\end{prop}

\begin{proof}
	Let $u=\phi(t)$. Then we can define $t = \phi^{-1}(u)$ by monotonicity.

	So
	\begin{align*}
		I(\omega) = \int_{\phi(a)}^{\phi(b)} f(\phi^{-1}(u)) \frac{e^{i\omega u}}{\phi'(\phi^{-1}(u))} du
	\end{align*}
\end{proof}

We now consider the case where $\phi(t)$ has a unique local max/min at $t=c, 1-c$, where  $\phi'(c) = 0, \phi''(c) > 0$
In the areas $a<t<c-\epsilon$ and $c+\epsilon<t<b$, our contribution is $\mathcal{O}(\frac{1}{\omega})$ as before. So, we consider
\begin{align*}
	\int_{c-\epsilon}^{c+\epsilon} f(t)e^{i\omega \phi(t)} dt &= \int_{c-\epsilon}^{c+\epsilon} f(t) e^{i\omega \left( \phi(c) + \frac{1}{2}\phi''(c)(t-c)^2 \right) } dt \\
	&\sim f(c) \int_{c-\epsilon}^{c+\omega} e^{i\frac{\omega}{2} \phi''(c) (t-c)^2} dt  \\
\end{align*}

We first consider $\phi''(c), \omega > 0$. Let  $s = \sqrt{\frac{\omega \phi''(c)}{2}}(t-c)$. Then
\begin{align*}
	\int_{c-\epsilon}^{c+\epsilon} f(t)e^{i\omega \phi(t)} dt &\sim \frac{f(c) e^{i\omega\phi(c)}}{\sqrt{\frac{\omega \phi''(c)}{2}}} \int_{-\epsilon \sqrt{\frac{\omega \phi''(c)}{2}} }^{\epsilon \sqrt{\frac{\omega \phi''(c)}{2}}} e^{is^2} ds \\
	&\sim \frac{f(c) e^{i\omega\phi(c)}}{\sqrt{\frac{\omega \phi''(c)}{2}}} \int_{-\infty}^{\infty} e^{is^2} ds \\
	&= \frac{2 f(c) e^{i\omega\phi(c)}}{\sqrt{\frac{\omega \phi''(c)}{2}}} \int_{0}^{\infty} e^{is^2} ds  \\
\end{align*}

We have found ourselves a Fresnel integral, $I = \int_{0}^{\infty} e^{ix s^2} ds = \frac{1}{2 }\sqrt{\frac{\pi}{x}} e^{i\frac{\pi}{4}} $. In general we might like to (for higher order expansions) consider $I_n = \int_{0}^{\infty} e^{ixs^{2n}}$. This can be done by considering wedges of angle $\frac{\pi}{2n}$.

So \[
	\int_{c-\epsilon}^{c+\epsilon} f(t) e^{i\omega \phi(t)} dt \sim \frac{f(c) e^{i\omega\phi(c)}\sqrt{2\pi} }{\sqrt{\omega \phi'(c)} } e^{i\frac{\pi}{4}}
\]

and in general, we get\[
	I(\omega) \sim f(c) e^{i\omega\phi(c)}\sqrt{\frac{2\pi}{|\omega \phi''(c)|}} e^{i\frac{\pi}{4} \sgn(\omega \phi''(c))} \text{ as } |\omega| \to \infty 
\]

As opposed to the Laplace's method, every local maxima or minima contributes. If our maxima occurs at the endpoint of the integral, we get a prefactor of $\frac{1}{2}$.

\begin{remark}
	If $c$ is a max/minx, with $\phi^{(2n)}\neq 0$ our first non-zero derivative, then we get that $I(\omega) \sim \mathcal{O}\left(\frac{1}{|\omega|^{\frac{1}{2n}}}  \right) $ 
\end{remark}

\begin{eg} Bessel Function of order $0$.

	\begin{align*}
		J_0(r) = \frac{1}{2\pi} \int_{0}^{2\pi} e^{ir \sin t} dt, \quad r\ge 0, \; r\to \infty
	\end{align*}
	For $\phi(t) = \sin t$, we have a maximum at $t = \frac{\pi}{2}$, and a minima at $t=\frac{3\pi}{2}$.

	At $\frac{\pi}{2}$, our contribution is
	\[
		\frac{1}{2\pi} \sqrt{\frac{2\pi}{r|-\sin(\frac{\pi}{2})}} e^{ir \sin \frac{\pi}{2} (-1)} = \frac{1}{\sqrt{2\pi r} } e^{ir} e^{-i\frac{\pi}{4}}
	,\]  and at $\frac{3\pi}{2}$ our contribution is similarly $\frac{1}{\sqrt{2\pi r} } e^{-ir} e^{i\frac{\pi}{4}}$.

	Then $J_0(r) \sim \sqrt{\frac{2}{\pi r}} \cos\left( r - \frac{\pi}{4} \right)  $.
\end{eg}

\subsection{Method of Steepest Descent}

We now wish to consider 
\[
	I(x) := \int_{C} g(z, x) dz
\] about some contour $C \subset \C$. We might like to try deforming our contour within a region of analyticity to some more convenient integral which is equal by Cauchy's Theorem.

We now consider the class of functions given by
\[
	I(x) = \int_{C} f(z) e^{x\phi(z)} dz
\] where $f, \phi$ are analytic in some region $D \supset C$.

Writing $\phi(z) = u(p, q) + iv(p,q)$, we have that $u_p = v_q$, $u_q = -v_p$, and so $\nabla u \cdot  \nabla v = 0$. Also, $\nabla u \neq  0 \iff \nabla v \neq 0 \iff \phi'(z) \neq 0$. We say then that $\phi$ is conformal at $z$.

\begin{eg}
	Take $\phi(z) = z^2 = p^2-q^2 + 2ipq$. Then $|e^{x\phi(z)}| = |e^{xu}|$ for real $x$.
\end{eg}

\subsubsection*{Ideal Case}

Ideally, we can deform $C \to \tilde{C}$ in a manner such that, for $e^{x\phi(z)} = e^{xu(p,q)} e^{iv(p,q)}$, $\tilde{C}$ is a level curve of $v$, i.e. $v(p,q) = v(p_0, q_0) = \text{const}$. 

In this case, \[
	I(x) = e^{iv(p_0,q_0)}\int_{\tilde{C}} f(z) e^{iu(p,q)} dz
\]

We can then parameterize $\tilde{C}$ as $z(t), \alpha \le t \le \beta$, and we can use Laplace's Method on 
\[
	I(x) = e^{ixv(p_0, q_0)} \int_{\tilde{C}} f(z(t)) e^{xu(p(t), q(t))} (\dot{p} + i \dot{q})dt
\]

\subsubsection*{Added Value}

In addition we have by Cauchy-Riemann that $\phi'(z_0) \neq 0 \iff \nabla u(p_0, q_0) \neq 0 \iff \nabla v(p_0, q_0) \neq 0$. Then if $phi'(z_0)$, we have a unique orthogonal intersection of curves with $u=\text{const}$ and $v = \text{const}$. The curve $v = \text{const}$ is in fact the steepest descent curve of $u$, that is along the direction of $\nabla u$.

\subsubsection*{Fixing Things}

Suppose $v(p_0, q_0) = v(p_1, q_1)$, where $C$ connects $z_0$ and $z_1$. In this case, we cannot deform to a curve $\tilde{C}$ with $v(p,q) = \text{const}$. 

\begin{idea}

We hope that by continuing two paths of $v=\text{const}$ from $z_0$ and $z_1$ to infinity, we can make it such that the connecting contour goes to 0, and we magically get the right answer (see example).
\end{idea}

Consider instead what happens at a point with $\phi'(z_0)$. Then $u$ (and $v$) is stationary at $(p_0, q_0)$. Now, $u,v$ are harmonic functions, so $u_{pp} = -u_{qq}$, and all interior stationary points must be saddle points. Then we can find a direction of steepest descent. 

\begin{eg} Saddle point

	Let $\phi(z) = iz^2$. Then $\phi'(z) = 2iz = 0 \iff z=0$.

	Now, $u = -2pq$, and $v = p^2 - q^2$.
	$v(0,0) = 0$. Consider the curve $v(p,q) = 0$. These are given by $p=q$ and $p=-q$.

	On these two curves,  $u= -2p^2$ or $u = 2p^2$. The former has a maximum at the origin, the latter has a minimum. Then to use steepest descent, we go along the curve $p=q$, where we can use Laplace's method.
\end{eg}

\begin{eg} (Bender and Orszage)

Let \[
	I(x) = \int_0^{1} e^{ixt^2} dt, \quad x\to \infty
\] (Fresnel integral at finite interval)

Then we expect, as $t^2$ is minimal at the origin, that $I(x) \sim \mathcal{O}\left(\frac{1}{\sqrt{x} }\right)$, with our leading contribution being from $t=0$. 

So, $\phi(z) = iz^2$, $f(z) = 1$, $C = [0,1]$. We have  $u = -2pq$, $v = p^2 - q^2$. Note that $v(0,0)  = 0 \neq v(1,0) = 1$.

By before, at $z =0$, our curve of steepest descent is $p=q$, where we go in to the UHP. Now, $\phi'(1) = 2i$, so there is a single line of steepest descent here. To have $v(p,q) = 1$, we have $p^2 - q^2 = 1$, so $p = \pm \sqrt{q^2 + 1} $. On this curve, we have $u = \mp 2q\sqrt{q^2 + 1} $. We pick the negative curve for steepest descent.  

We consider truncating at $q=T$, connecting the two curves with a segment of $q=T$.  

So we let $\Gamma_2(T) = \{z + s + iT: T \le s \le \sqrt{T^2 +1} \} $.

\begin{align*}
	\left| \int_{\Gamma_2(T)} e^{ixz^2} dz \right| &= \left| \int_{T}^{\sqrt{T^2 + 1} } e^{ix \left( s^2 -T^2  + 2isT \right) } ds \right| \\
	&\le  \left| \int_T^{\sqrt{T^2 + 1} } e^{-2sT} ds  \right| \\
	&= \frac{1}{2xT} \left( e^{-2xT^2} - e^{-2xT\sqrt{T^2 + 1} } \right)  \\
	& \to 0 \text{ as } T \to \infty
\end{align*}

Let $\Gamma_1(T)$ be our curve of steepest descent,  $p=q$.

\begin{align*}
	\int_{\Gamma_1}  e^{ixz^2} dz &= \int_0^{\infty} e^{-2q^2 x} (1+i) dq  \\
	&= \frac{1}{2} \sqrt{\frac{\pi}{x}} e^{i\frac{\pi}{4}} \\
\end{align*}

Finally, we let $\Gamma_3$ be our right curve of steepest descent, $p = \sqrt{q^2 + 1} $.

\begin{align*}
	\int_{-\Gamma_3} e^{ixz^2} dz &= \int_{0}^{\infty} \left( \frac{q}{\sqrt{1+q^2} } + 1 \right) e^{ix}e^{-2q\sqrt{1+q^2} } dq\\
\end{align*}
Let $s = 2q \sqrt{1+q^2} $. Then $i z^2 = i - s, z^2 = 1 + is$.

So
\begin{align*}
	\int_{-\Gamma_3} e^{ixz^2} dz &= \int_{0}^{\infty} \frac{e^{(i-s)x} i}{2(1+is)^{\frac{1}{2}}} ds \\
	&= \frac{i}{2} e^{ix} \int_0^{\infty} \frac{e^{-sx}}{(1+is)^{\frac{1}{2}}} \\
\end{align*}

Now, 
\[
	\frac{1}{(1+is)^{\frac{1}{2}}} = \sum_{n=0}^{\infty} (-i)^{n} \frac{s^{n}}{n!} \frac{\Gamma\left(n+\frac{1}{2}\right)}{\Gamma \left( \frac{1}{2} \right)}
,\] 
And so by Watson's Lemma,
\begin{align*}
	\int_{-\Gamma_3} e^{ixz^2} dz &\sim \frac{i}{2} e^{ix} \sum_{n=0}^{\infty} (-i)^{n} \frac{\Gamma \left(n + \frac{1}{2} \right)}{\Gamma \left( \frac{1}{2} \right) n! } \frac{\Gamma \left( n+1 \right) }{x^{n+1}} \\
	&= \frac{e^{i\frac{\pi}{2}}}{2} \sum_{n=0}^{\infty} (-i)^{n} \frac{\Gamma \left( n+\frac{1}{2} \right) }{\Gamma\left( \frac{1}{2} \right) } \frac{1}{x^{n+1}}
\end{align*}

And \[
	I(x) \sim \int_{\Gamma_1} - \int_{-\Gamma_3}
\] 
\end{eg}

We go back now to our general form
\[
	\int_{C} f(z) e^{x(u+iv)} dz
\]

We seek to deform $C$ to reach a saddle point, as there we have multiple curves of steepest descent, giving then the major contribution to the integral from the saddle point.

Suppose $C$ in fact starts from a saddle point and is of steepest descent for u. Then $\phi'(z_0) = 0$. Assume that $\phi''(z_0) \neq 0$.

\begin{align*}
	\int_{C} f(z) e^{x\phi(z)} dz = e^{ixv_0} \int_{C} f(z) e^{xu(z)} dz
\end{align*}
As $C$ is a curve of steepest descent, we can happily consider just a small region about $z_0$ for our majority contribution.

So
 \begin{align*}
	 \int_{C} f(z) e^{x\phi(z)} dz \sim f(z_0) e^{x\phi(z_0)}\int_{C_\epsilon} e^{\frac{1}{2} \phi''(z_0) (z-z_0)^2} dz
\end{align*}

Let $z(t) = z_0 + r(t) e^{i\theta(t)}$. Then $\phi''(z_0) (z-z_0)^2 = \phi''(z_0) r^2 e^{2i\theta(t)}$, and $\phi''(z_0) = |\phi''(z_0)| e^{i\alpha}$. $\Im (\phi(z) - \phi(z_0)) =0 $ on $C_{\epsilon}$, as $v$ is constant. So, writing  $\phi''(z_0) (z-z_0)^2 = |\phi''(z_0)| r^2 e^{i(2\theta + \alpha)}$, we have that $\sin (2\theta + \alpha) = 0$. Also, $\Re (\phi(z) - \phi(z_0)) <0$ for $z \neq z_0$ on $C_{\epsilon}$. So $\cos(2\theta + \alpha) < 0$, and so  $2\theta + \alpha = (2k+1)\pi$, $k=0,1$.

Then
 \begin{align*}
	 I(x) &\sim f(z_0) e^{x\phi(z_0)} \int_{C_{\epsilon}} e^{\frac{x}{2} |\phi''(z_0)| r^2 e^{i(2\theta + \alpha)}} d(re^{i\theta}) \\
	 &\sim  f(z_0) e^{x\phi(z_0)} e^{i\theta}\int_{0}^{\epsilon} e^{\frac{x}{2} |\phi''(z_0)| r^2 e^{i(2\theta + \alpha)}} dr \\
\end{align*}
Let $s = -i \left( \frac{|\phi''(z_0)| x}{2} \right)^{\frac{1}{2}} re^{i(\theta + \frac{\alpha}{2})} $ 

Then 
\begin{align*}
	I(x) &\sim f(z_0) e^{i\theta} e^{x\phi(z_0)} i \left( \frac{2}{x |\phi''(z_0)|} \right)^{\frac{1}{2}} e^{-i\frac{\alpha}{2}} e^{-i\theta} \int_{0}^{\infty} e^{-s^2} ds \\
	&\sim  \frac{i}{2} \left( \frac{2\pi}{x\phi''(z_0)} \right)^{\frac{1}{2}} e^{x\phi(z_0)} f(z_0)
\end{align*} as $\phi''(z_0) = |\phi''(z_0)| e^{i\alpha}$. 
\end{document}
